\newif\ifdraft \drafttrue
%% For double-blind review submission, w/o CCS and ACM Reference (max submission space)
\documentclass[acmsmall,review,anonymous]{acmart}\settopmatter{printfolios=true,printccs=false,printacmref=false}
%% For double-blind review submission, w/ CCS and ACM Reference
%\documentclass[acmsmall,review,anonymous]{acmart}\settopmatter{printfolios=true}
%% For single-blind review submission, w/o CCS and ACM Reference (max submission space)
%\documentclass[acmsmall,review]{acmart}\settopmatter{printfolios=true,printccs=false,printacmref=false}
%% For single-blind review submission, w/ CCS and ACM Reference
%\documentclass[acmsmall,review]{acmart}\settopmatter{printfolios=true}
%% For final camera-ready submission, w/ required CCS and ACM Reference
%\documentclass[acmsmall]{acmart}\settopmatter{}


%% Journal information
%% Supplied to authors by publisher for camera-ready submission;
%% use defaults for review submission.
\acmJournal{PACMPL}
\acmVolume{1}
\acmNumber{CONF} % CONF = POPL or ICFP or OOPSLA
\acmArticle{1}
\acmYear{2018}
\acmMonth{1}
\acmDOI{} % \acmDOI{10.1145/nnnnnnn.nnnnnnn}
\startPage{1}

%% Copyright information
%% Supplied to authors (based on authors' rights management selection;
%% see authors.acm.org) by publisher for camera-ready submission;
%% use 'none' for review submission.
\setcopyright{none}
%\setcopyright{acmcopyright}
%\setcopyright{acmlicensed}
%\setcopyright{rightsretained}
%\copyrightyear{2018}           %% If different from \acmYear

%% Bibliography style
\bibliographystyle{ACM-Reference-Format}
%% Citation style
%% Note: author/year citations are required for papers published as an
%% issue of PACMPL.
\citestyle{acmauthoryear}   %% For author/year citations


%%%%%%%%%%%%%%%%%%%%%%%%%%%%%%%%%%%%%%%%%%%%%%%%%%%%%%%%%%%%%%%%%%%%%%
%% Note: Authors migrating a paper from PACMPL format to traditional
%% SIGPLAN proceedings format must update the '\documentclass' and
%% topmatter commands above; see 'acmart-sigplanproc-template.tex'.
%%%%%%%%%%%%%%%%%%%%%%%%%%%%%%%%%%%%%%%%%%%%%%%%%%%%%%%%%%%%%%%%%%%%%%


%% Some recommended packages.
\usepackage{booktabs}   %% For formal tables:
                        %% http://ctan.org/pkg/booktabs
\usepackage{subcaption} %% For complex figures with subfigures/subcaptions
                        %% http://ctan.org/pkg/subcaption


\newcommand\tuple[1]{\left\langle {#1} \right\rangle}
\newcommand\states{Q}
\newcommand\st{q}
\newcommand\vars{V}
\newcommand\var{X}
\newcommand\init{I}
\newcommand\flow{F}
\newcommand\inv{\mathit{inv}}
\newcommand\cmd{c}
\newcommand\set[1]{\left\{{#1}\right\}}
\newcommand\REAL{\mathbb{R}}
\newcommand\todo[1]{\textbf{TODO: {#1}}}
\newcommand\POWER{\mathcal{p}}
\newcommand\power[1]{2^{#1}}
\newcommand\ra{\rightarrow}
\newcommand\VALUATIONS{\Sigma}
\newcommand\valuation{\sigma}
\DeclareMathOperator{\COL}{{:}}
\DeclareMathOperator{\DEFEQ}{{:=}}
\newcommand\der[1]{\dot{#1}}
% \newcommand\prime[1]{{#1}'}
\newcommand\fml{\varphi}
% \DeclareMathOperator{\notmodels}{{\not\models}}
\newcommand\NONNEGREAL{\REAL^{{} \ge 0}}
\newcommand\setfml{\mathbf{Fml}}
\newcommand\trans{\delta}
\newcommand\sol{x}
\newcommand\DDT[1]{\frac{d{#1}}{dt}}
\newcommand\ODE{\mathcal{D}}
\newcommand\sem[1]{\left\llbracket{#1}\right\rrbracket}

\newcommand\semtrue{\top}

\newcommand\CONTIREACH[1]{\operatorname{\rightarrow_{\ODE,{#1}}}}

%%% Local Variables:
%%% mode: japanese-latex
%%% TeX-master: main.tex
%%% End:


\begin{document}

%% Title information
\title[]{Property-Directed Reachability for Hybrid Systems with ODE Reachability Predicates}
% \titlenote{with title note}
% \subtitle{Proof tree decoder with deep neural networks}
% \subtitlenote{with subtitle note}


%% Author information
%% Contents and number of authors suppressed with 'anonymous'.
%% Each author should be introduced by \author, followed by
%% \authornote (optional), \orcid (optional), \affiliation, and
%% \email.
%% An author may have multiple affiliations and/or emails; repeat the
%% appropriate command.
%% Many elements are not rendered, but should be provided for metadata
%% extraction tools.

%% Author with single affiliation.
\author{First1 Last1}
\authornote{with author1 note}          %% \authornote is optional;
                                        %% can be repeated if necessary
\orcid{nnnn-nnnn-nnnn-nnnn}             %% \orcid is optional
\affiliation{
  \position{Position1}
  \department{Department1}              %% \department is recommended
  \institution{Institution1}            %% \institution is required
  \streetaddress{Street1 Address1}
  \city{City1}
  \state{State1}
  \postcode{Post-Code1}
  \country{Country1}                    %% \country is recommended
}
\email{first1.last1@inst1.edu}          %% \email is recommended

%% Author with two affiliations and emails.
\author{First2 Last2}
\authornote{with author2 note}          %% \authornote is optional;
                                        %% can be repeated if necessary
\orcid{nnnn-nnnn-nnnn-nnnn}             %% \orcid is optional
\affiliation{
  \position{Position2a}
  \department{Department2a}             %% \department is recommended
  \institution{Institution2a}           %% \institution is required
  \streetaddress{Street2a Address2a}
  \city{City2a}
  \state{State2a}
  \postcode{Post-Code2a}
  \country{Country2a}                   %% \country is recommended
}
\email{first2.last2@inst2a.com}         %% \email is recommended
\affiliation{
  \position{Position2b}
  \department{Department2b}             %% \department is recommended
  \institution{Institution2b}           %% \institution is required
  \streetaddress{Street3b Address2b}
  \city{City2b}
  \state{State2b}
  \postcode{Post-Code2b}
  \country{Country2b}                   %% \country is recommended
}
\email{first2.last2@inst2b.org}         %% \email is recommended


%% Abstract
%% Note: \begin{abstract}...\end{abstract} environment must come
%% before \maketitle command

\begin{abstract}
\end{abstract}



%% 2012 ACM Computing Classification System (CSS) concepts
%% Generate at 'http://dl.acm.org/ccs/ccs.cfm'.
\begin{CCSXML}
<ccs2012>
<concept>
<concept_id>10011007.10011006.10011008</concept_id>
<concept_desc>Software and its engineering~General programming languages</concept_desc>
<concept_significance>500</concept_significance>
</concept>
<concept>
<concept_id>10003456.10003457.10003521.10003525</concept_id>
<concept_desc>Social and professional topics~History of programming languages</concept_desc>
<concept_significance>300</concept_significance>
</concept>
</ccs2012>
\end{CCSXML}

\ccsdesc[500]{Software and its engineering~General programming languages}
\ccsdesc[300]{Social and professional topics~History of programming languages}
%% End of generated code


%% Keywords
%% comma separated list
\keywords{Deep Learning, Deep Neural Networks, Automatic Theorem Proving}


%% \maketitle
%% Note: \maketitle command must come after title commands, author
%% commands, abstract environment, Computing Classification System
%% environment and commands, and keywords command.
\maketitle

% \input{sec/intro}
% \input{sec/logic}
% \input{sec/background}
% \documentclass[a4paper]{jarticle}

\usepackage{amsmath,amsthm,amssymb}
\usepackage{stmaryrd}

\theoremstyle{definition}
\newtheorem{theorem}{定理}
\newtheorem{lemma}{補題}


\newcommand\tuple[1]{\left\langle {#1} \right\rangle}
\newcommand\states{Q}
\newcommand\st{q}
\newcommand\vars{V}
\newcommand\var{X}
\newcommand\init{I}
\newcommand\flow{F}
\newcommand\inv{\mathit{inv}}
\newcommand\cmd{c}
\newcommand\set[1]{\left\{{#1}\right\}}
\newcommand\REAL{\mathbb{R}}
\newcommand\todo[1]{\textbf{TODO: {#1}}}
\newcommand\POWER{\mathcal{p}}
\newcommand\power[1]{2^{#1}}
\newcommand\ra{\rightarrow}
\newcommand\VALUATIONS{\Sigma}
\newcommand\valuation{\sigma}
\DeclareMathOperator{\COL}{{:}}
\DeclareMathOperator{\DEFEQ}{{:=}}
\newcommand\der[1]{\dot{#1}}
% \newcommand\prime[1]{{#1}'}
\newcommand\fml{\varphi}
% \DeclareMathOperator{\notmodels}{{\not\models}}
\newcommand\NONNEGREAL{\REAL^{{} \ge 0}}
\newcommand\setfml{\mathbf{Fml}}
\newcommand\trans{\delta}
\newcommand\sol{x}
\newcommand\DDT[1]{\frac{d{#1}}{dt}}
\newcommand\ODE{\mathcal{D}}
\newcommand\sem[1]{\left\llbracket{#1}\right\rrbracket}

\newcommand\semtrue{\top}

\newcommand\CONTIREACH[1]{\operatorname{\rightarrow_{\ODE,{#1}}}}

%%% Local Variables:
%%% mode: japanese-latex
%%% TeX-master: main.tex
%%% End:


\begin{document}

\begin{theorem}
  $\CONSISTENT$はHybridPDRの全ての規則の不変条件となっている
\end{theorem}

\begin{proof}
  GPDRの証明に準ずる
\end{proof}

\begin{lemma} \label{lem:1}
  任意の$n \in \mathbb{N}, \valuation \in \VALUATIONS, \st \in \states$について
  $(\valuation, \st) \in \REACH{n}$ならば$\valuation \models R_n$となる
\end{lemma}

\begin{proof}
  $n$に関する数学的帰納法で示す.
  $n=0$のとき,$R_0 = \predtrans(\FALSE) = \fml_0, \REACH{0} = \{(\valuation, \st) | \valuation \models \fml_0 \}$より成り立つ.

  $n=k$のとき成り立つと仮定すると,帰納法の仮定と$\mathit{Reach}$の定義より
  $(\valuation,\st) \in \REACH{k} \Rightarrow [\predtrans(R_k)]_{\valuation}$
  が成り立つ.
  $\CONSISTENT$が成り立つことより$\predtrans(R_k) \Rightarrow R_{k+1}$なので,
  これらより$(\valuation,\st) \in \REACH{k+1} \Rightarrow [R_{k+1}]_\valuation$
  が成り立つ.
\end{proof}

\begin{lemma} \label{lem:2}
  任意の$n \in \mathbb{N}, \valuation \in \VALUATIONS, \st \in \states$について,
  $(\valuation,\st) \in \REACH{n}$のとき,またそのときに限り,
  ある$n$-length run $(\valuation_0,\st_0) \dots (\valuation_n, \st_n)$が存在し,
  $(\valuation, \st) = (\valuation_n, \st_n)$を満たすか,
  ある$k < n$が存在し,$q_k = q$となる$q_k$に対し,$\valuation_k \CONTIREACH{F(q_k)}{ inv(q_k)} \valuation$を満たす.
\end{lemma}

\begin{proof}
  $n$に関する数学的帰納法で証明する.$n=0$のときは自明.
  $n=m$で成り立つ時,
  \begin{itemize}
  \item ($\Rightarrow$)を示す.
  \item ($\Leftarrow$)を示す.
    $(\valuation,\st) = (\valuation_{m+1}, \st_{m+1})$のとき,帰納法の仮定より
    $(\valuation_m,\st_m) \in \REACH{m}$であること,
    runの定義よりある$\sigma_m'$が存在して,$\sigma_m \CONTIREACHD{inv(\st_m)} \valuation_m', \valuation_m' \models \GUARD_\delta(\st_m,\st_{m+1}), \valuation_{m+1} = \sem{\CMD_\delta(\st_m,\st_{m+1})}$が成り立つことと$\mathit{Reach}$の定義より,
    $(\valuation,\st) \in \REACH{m+1}$が成り立つ.

    $(\valuation,\st) \not= (\valuation_{m+1},\st_{m+1})$のとき,
    仮定より$k<m+1$で$q_k=q,\valuation_k \CONTIREACHD{inv(q_k)} \valuation$となる
    $k$が存在する.
    $k < m$のとき,帰納法の仮定より$(\valuation,\st) \in \REACH{m}$となり,
    $\REACH{m} \subseteq \REACH{m+1}$より,$(\valuation,\st) \in \REACH{m+1}$.
    $k = m$のとき,帰納法の仮定より$(\valuation_m,\st_m) \in \REACH{m}$なので,
    これと仮定と$\mathit{Reach}$の定義より,$(\valuation, \st) \in \REACH{m+1}$.
  \end{itemize}
\end{proof}

\begin{theorem}
  HybridPDRがunsafeならば,unsafeな状態を通るrunが存在する.
\end{theorem}

\begin{proof}
  補題\ref{lem:1},\ref{lem:2}より直ちに示される.
\end{proof}

\end{document}

% % \input{sec/statics}
% \input{sec/model}
% \input{sec/exp}
% \input{sec/relwork}
% \input{sec/conclusion}

% \input{sec/onFinitenessNormal}

\section{Preliminary}

We write $\REAL$ for the set of reals and $\NONNEGREAL$ for the set of
nonnegative reals.  We fix a finite set
$\vars := \set{\var_1,\dots,\var_N}$ of \emph{continuous
  variables} or simply \emph{variables}.  They denote a real number
that evolves over time: A variable is bound to a function from
$\NONNEGREAL$ to $\REAL$.
% We write $\prime{\vars}$ for the set
% $\set{\prime{\var_0},\prime{\var_1},\dots,\prime{\var_N}}$ of the
% \emph{primed} variables.
We also introduce a set
$\der{\vars} \DEFEQ
\set{\der{\var_1},\dots,\der{\var_N}}$ of the variables
that denote the time derivatives of the variables in $\vars$.

\todo{Only variables in $\vars$ are allowed to be free.}  We write
$\setfml$ for the set of logical formulas over $\vars$; its element is
ranged over by $\fml$.  \todo{Explain the logic.} We call an element
of the set $\VALUATIONS \DEFEQ \vars \ra \REAL$ a \emph{valuation};
they are represented by metavariable $\valuation$.  We write
$\valuation \models \fml$ if $\valuation$ is a model of $\fml$;
$\valuation \not\models \fml$ if $\valuation \models \fml$ does not
hold; $\models \fml$ if $\valuation \models \fml$ for any
$\valuation$; and $\not\models \fml$ if there exists
$\valuation \models \fml$.  \todo{Definition of $\sem{\fml}$.}

\todo{We may want to limit the end time $T$.}  An ODE is a set of
equations of the form
$\der{\var_1} = f_1(\var_1,\dots,\var_N), \dots, \der{\var_N} =
f_N(\var_1,\dots,\var_N)$, where $f_1,\dots,f_N : \REAL^N \ra \REAL$
determines the derivatives of each variable.  Given a valuation
$\valuation_0$ on $\var_1,\dots,\var_N$ that determines their initial
values and a formula $\fml$, a set of $C^1$ functions
$\sol_1,\dots,\sol_N \COL \NONNEGREAL \ra \REAL$ that satisfy the
following conditions are said to be a \emph{solution} of ODE
$\der{\var_1} = f_1(\var_1,\dots,\var_N), \dots, \der{\var_N} =
f_N(\var_1,\dots,\var_N)$ that \emph{keeps satisfying} $\fml$:
\begin{itemize}
\item
  $\sol_1(0) = \valuation(\var_1),\dots,\sol_N(0) =
  \valuation(\var_N)$,
\item for any $t \in \NONNEGREAL$ and $i \in \set{1,\dots,N}$,
  $\DDT{\sol_i}(t) = f_i(\sol_1(t),\dots,\sol_N(t))$, and
\item for any $t \in \NONNEGREAL$,
  $\set{\var_1 \mapsto \sol_1(t),\dots,\var_N \mapsto \sol_N(t)}
  \models \fml$.
\end{itemize}
\todo{Check whether this definition of ODE is correct.}  In this
paper, we focus on ODE where each $f_i$ is Lipschitz continuous.  We
use metavariable $\ODE$ for an ODE.  Intuitively, a solution of an ODE
$\ODE$ keeps satisfying $\fml$ if its trajectory stays in
$\sem{\fml}$.  The standard notion of the solution of an ODE is
obtained by setting $\fml$ to $\semtrue$.  We write
$\valuation_0 \CONTIREACH{\fml} \valuation$ if $\valuation$ is
reachable form $\valuation_0$ if there is a solution of $\ODE$ that
keeps satisfying $\fml$: There is $T \in \NONNEGREAL$ and a solution
$\set{\sol_1,\dots,\sol_N}$ of $\ODE$ from the initial valuation
$\valuation_0$ such that $\valuation(\var_i) = \sol_i(T)$.

We consider the hybrid systems specified by \emph{hybrid automata}
whose definition is as follows.
\begin{definition}[Hybrid automaton]
  A \emph{hybrid automaton} is a tuple
  $\tuple{\states,\init,\flow,\inv,\trans}$.  Here,
  $\states = \set{\st_0,\st_1,\st_2,\dots}$ is a set of
  \emph{locations}.  $\init \in \states \times \setfml$ represents the
  initial state.

  $\flow \in \states \times \vars \ra f(\vec\vars)$,
  $\inv \in \states \ra \power{\REAL^n}$,
\end{definition}

% Edit here.

\section{Introduction}

\section{Preliminaly}

\subsection{Syntax}

\section{Hybrid automaton}

% Defintion

%%%%% Assume that DE is Lipsitz continuous.

% Definition of run.

\section{Standard PDR}

\section{Hybrid PDR}

\begin{align*}
  \begin{array}{lll}
    \Initialize & \Longrightarrow \System{\epsilon}{[N \Subst 0, R_0 \Subst \PredTrans(false)]} &  \\
    \Valid & \System{M}{A} \Longrightarrow \Valid & \If \models R_{i-1} \subseteq R_{i}, i < N \\
    \Unfold & \System{M}{A} \Longrightarrow \System{\epsilon}{A[R_{N+1} \Subst true, N \Subst N+1]} & \If \models R_N \implies S 
  \end{array}
\end{align*}

\section{Correctness}

\section{Experiments}

\section{Related work}

\section{Conclusion}


% \input{sec/onFinitenessNormal}

%% Acknowledgments
\begin{acks}                            %% acks environment is optional
                                        %% contents suppressed with 'anonymous'
  %% Commands \grantsponsor{<sponsorID>}{<name>}{<url>} and
  %% \grantnum[<url>]{<sponsorID>}{<number>} should be used to
  %% acknowledge financial support and will be used by metadata
  %% extraction tools.
  This material is based upon work supported by the
  \grantsponsor{GS100000001}{National Science
    Foundation}{http://dx.doi.org/10.13039/100000001} under Grant
  No.~\grantnum{GS100000001}{nnnnnnn} and Grant
  No.~\grantnum{GS100000001}{mmmmmmm}.  Any opinions, findings, and
  conclusions or recommendations expressed in this material are those
  of the author and do not necessarily reflect the views of the
  National Science Foundation.
\end{acks}


%% Bibliography
\bibliography{main}


%% Appendix
\appendix
\section{Appendix}

Text of appendix \ldots

\end{document}

%%% Local Variables:
%%% mode: japanese-latex
%%% TeX-master: t
%%% End:

\documentclass[a4paper]{jarticle}

\usepackage{amsmath,amsthm,amssymb}
\usepackage{stmaryrd}

\theoremstyle{definition}
\newtheorem{theorem}{定理}
\newtheorem{lemma}{補題}

\newcommand\MODEL{\mathcal{M}}
\newcommand\PredTrans{\mathcal{F}}
\newcommand\TRANS{\mathcal{T}}
\newcommand\DYNAMICS{\mathcal{D}}

\newcommand\Var{\mathbf{Var}}
\newcommand\PVar{\mathbf{PVar}}

\newcommand\Prim{p}
\newcommand\Clause{\varphi}

\newcommand\Initialize{\mathbf{Initialize}}
\newcommand\Valid{\mathbf{Valid}}
\newcommand\Unfold{\mathbf{Unfold}}
\newcommand\Induction{\mathbf{Induction}}
\newcommand\Candidate{\mathbf{Candidate}}
\newcommand\Decide{\mathbf{Decide}}
\newcommand\Model{\mathbf{Model}}
\newcommand\Conflict{\mathbf{Conflict}}

\newcommand\ContTrans[2]{\longleftarrow^{#1 \> #2}}

\newcommand\System[2]{#1\>||\>#2}

\newcommand\Subst{\leftarrow}

\newcommand\If{\mathbf{if}\>}

\newcommand\Tuple[1]{\left\langle{#1}\right\rangle}


\begin{document}

\begin{theorem}
  $\CONSISTENT$はHybridPDRの全ての規則の不変条件となっている
\end{theorem}

\begin{proof}
  GPDRの証明に準ずる
\end{proof}

\begin{lemma} \label{lem:1}
  任意の$n \in \mathbb{N}, \valuation \in \VALUATIONS, \st \in \states$について
  $(\valuation, \st) \in \REACH{n}$ならば$\valuation \models R_n$となる
\end{lemma}

\begin{proof}
  $n$に関する数学的帰納法で示す.
  $n=0$のとき,$R_0 = \predtrans(\FALSE) = \fml_0, \REACH{0} = \{(\valuation, \st) | \valuation \models \fml_0 \}$より成り立つ.

  $n=k$のとき成り立つと仮定すると,帰納法の仮定と$\mathit{Reach}$の定義より
  $(\valuation,\st) \in \REACH{k} \Rightarrow [\predtrans(R_k)]_{\valuation}$
  が成り立つ.
  $\CONSISTENT$が成り立つことより$\predtrans(R_k) \Rightarrow R_{k+1}$なので,
  これらより$(\valuation,\st) \in \REACH{k+1} \Rightarrow [R_{k+1}]_\valuation$
  が成り立つ.
\end{proof}

\begin{lemma} \label{lem:2}
  任意の$n \in \mathbb{N}, \valuation \in \VALUATIONS, \st \in \states$について,
  $(\valuation,\st) \in \REACH{n}$のとき,またそのときに限り,
  ある$n$-length run $(\valuation_0,\st_0) \dots (\valuation_n, \st_n)$が存在し,
  $(\valuation, \st) = (\valuation_n, \st_n)$を満たすか,
  ある$k < n$が存在し,$q_k = q$となる$q_k$に対し,$\valuation_k \CONTIREACH{F(q_k)}{ inv(q_k)} \valuation$を満たす.
\end{lemma}

\begin{proof}
  $n$に関する数学的帰納法で証明する.$n=0$のときは自明.
  $n=m$で成り立つ時,
  \begin{itemize}
  \item ($\Rightarrow$)を示す.
  \item ($\Leftarrow$)を示す.
    $(\valuation,\st) = (\valuation_{m+1}, \st_{m+1})$のとき,帰納法の仮定より
    $(\valuation_m,\st_m) \in \REACH{m}$であること,
    runの定義よりある$\sigma_m'$が存在して,$\sigma_m \CONTIREACHD{inv(\st_m)} \valuation_m', \valuation_m' \models \GUARD_\delta(\st_m,\st_{m+1}), \valuation_{m+1} = \sem{\CMD_\delta(\st_m,\st_{m+1})}$が成り立つことと$\mathit{Reach}$の定義より,
    $(\valuation,\st) \in \REACH{m+1}$が成り立つ.

    $(\valuation,\st) \not= (\valuation_{m+1},\st_{m+1})$のとき,
    仮定より$k<m+1$で$q_k=q,\valuation_k \CONTIREACHD{inv(q_k)} \valuation$となる
    $k$が存在する.
    $k < m$のとき,帰納法の仮定より$(\valuation,\st) \in \REACH{m}$となり,
    $\REACH{m} \subseteq \REACH{m+1}$より,$(\valuation,\st) \in \REACH{m+1}$.
    $k = m$のとき,帰納法の仮定より$(\valuation_m,\st_m) \in \REACH{m}$なので,
    これと仮定と$\mathit{Reach}$の定義より,$(\valuation, \st) \in \REACH{m+1}$.
  \end{itemize}
\end{proof}

\begin{theorem}
  HybridPDRがunsafeならば,unsafeな状態を通るrunが存在する.
\end{theorem}

\begin{proof}
  補題\ref{lem:1},\ref{lem:2}より直ちに示される.
\end{proof}

\end{document}
